Granary is designed to comprehensively instrument kernel modules. We first evaluated the performance of Granary framework alone by running it with different kernel modules. We ran the experiment on a desktop equipped with an Intel\textregistered\ Core\texttrademark\ i7-860 2.80 GHz CPU, 8GB memory, and an Intel 82578DM Gigabit Ethernet card. We used IOzone filesystem benchmark to exercise ext3 module while running it with Granary. Iozone is a useful benchmark for testing data throughput with number of different access pattern. we avoided bottleneck for disk I/O by partitioning main memory to load ext3 module. To compare the throughput of Granary with existing kernel instrumentation framework, we modified DRK to instrument only kernel modules. We implemented similar attach/detach mechanism for DRK at module interface. The framework uses similar kernel wrappers for fast attach and page protection as fall back mechanism to gain control. For initial evaluation of Granary, we developed null policy client and enabled some of the optimizations.

%wrapping mechanism for DRK to attach and detach the framework at module interface.

%and jbd modules while running them under Granary.   


%To evaluate our system with existing kernel instrumentation framework, we modified DRK to instrument only kernel modules. We implemented similar wrapping mechanism for DRK to attach and detach the framework at module interface. We checked the correctness of our system by running different filesystem and network modules(ext2/ext3, xfs, e1000e etc) under their control. To stress test the filesystem modules we used filesystem benchmark IOzone and ran the modules with null instrumentation client. null instrumentation client was used to avoid the overhead of instrumentation. We ran our tests on a desktop equipped with an Intel\textregistered\ Core\texttrademark\ i7-860 2.80 GHz CPU, 8GB memory, and an Intel 82578DM Gigabit Ethernet card.  