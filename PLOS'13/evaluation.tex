

%\begin{figure}[ht!]
% \subfloat{
%\epsfig{file=diagrams/iozone-write-workload.eps,width=3.4in}
% }
\ORIGcaption{\label{fig:iozone-write-workload}Writer procress throughput under different workloads}
\end{figure}


%\begin{table}[thp!]
%\ORIGcaption{Iozone Throughput for different filesystem operations(KBytes/Sec)}
% title of Table
%\centering
% used for centering table
%\begin{tabular}{l r r r}
% centered columns (4 columns)
%\hline\hline
%inserts double horizontal lines
%Workload & Native & Granary & Modified Drk  \\ [0.5ex]
% inserts table
%heading
%\hline
% inserts single horizontal line
%Write & 184669.69 & 152223.03 & 55134.09 \\
% inserting body of the table
%Read & 1577591.69 & 1648645.02 & 1421037.96 \\
%Random Write  & 552380.48 & 412069.48 & 162847.34 \\
%Random Read & 1644573.29 & 1597321.54 & 1636472.14 \\

%Mixed & 1168804.11 & 929383.54 & 915196.76 \\[1ex]
% [1ex] adds vertical space
%\hline
%inserts single line
%\end{tabular}
%\label{table:nonlin}
% is used to refer this table in the text
%\end{table}

%wrapping mechanism for DRK to attach and detach the framework at module interface.

%and jbd modules while running them under Granary.   


%To evaluate our system with existing kernel instrumentation framework, we modified DRK to instrument only kernel modules. We implemented similar wrapping mechanism for DRK to attach and detach the framework at module interface. We checked the correctness of our system by running different filesystem and network modules(ext2/ext3, xfs, e1000e etc) under their control. To stress test the filesystem modules we used filesystem benchmark IOzone and ran the modules with null instrumentation client. null instrumentation client was used to avoid the overhead of instrumentation. We ran our tests on a desktop equipped with an Intel\textregistered\ Core\texttrademark\ i7-860 2.80 GHz CPU, 8GB memory, and an Intel 82578DM Gigabit Ethernet card.  