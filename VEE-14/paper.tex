\documentclass[preprint]{sigplanconf}
\usepackage{cleveref,amsmath,chngcntr}
\usepackage{textcomp} % for tilde
\usepackage{paralist} % for in-paragraph lists
\usepackage{url}
\usepackage{graphicx}
\usepackage{multicol,multirow}
\usepackage{tabularx}

% For drawing FSMs
%\usepackage{tikz}
%\usetikzlibrary{arrows,automata}

% Used for code snippets
\usepackage{listings,courier}
\usepackage{subfig,epsfig}
\DeclareCaptionType{copyrightbox}
\usepackage{epsfig}


% Settings on code listings.
\lstset{language=C,
		xleftmargin=0pt,
		xrightmargin=0pt,
		framexbottommargin=0pt,
        framextopmargin=0pt,
        framesep=0pt}
\usepackage{enumitem}

% Add listing language for assembly
\lstdefinelanguage
   [x64]{Assembler}     % add a "x64" dialect of Assembler
   [x86masm]{Assembler} % based on the "x86masm" dialect
   % with these extra keywords:
   {morekeywords={CDQE,CQO,CMPSQ,CMPXCHG16B,JRCXZ,LODSQ,MOVSXD, %
                  POPFQ,PUSHFQ,SCASQ,STOSQ,IRETQ,RDTSCP,SWAPGS, %
                  rax,rdx,rcx,rbx,rsi,rdi,rsp,rbp, %
                  r8,r8d,r8w,r8b,r9,r9d,r9w,r9b}} % etc.


% For leaving some comments in the draft.
\newcommand{\comment}[1]{}

% Customize cleverref
\crefname{section}{Section}{Sections}

\begin{document}

%make title bold and 14 pt font (Latex default is non-bold, 16 pt)
\title{Flexible Binary Instrumentation for the Kernel}

%for single author (just remove % characters)
\authorinfo{Peter Goodman \and Akshay Kumar \and Angela Demke Brown \and Ashvin Goel}
{University of Toronto}{}
% end authorinfo

\maketitle
\subsection*{Abstract}
%Kernel modules extend the functionality of operating systems. They represent the bulk of new kernel code in development and contain the majority of new kernel bugs. Unfortunately, analyzing and debugging modules is very challenging. Static analysis of module source code is difficult because of the tight interaction between modules and the kernel. Some modules, however, are only distributed in a binary format, which makes static analysis intractable. Existing dynamic analysis tools capable of instrumenting modules are not suited toward module analysis. They either impose unnecessary performance overheads, lack flexibility and are too coarse-grained, or do not provide a means of understanding the ``big picture'' view of what an arbitrary module does as it executes.

%We created Granary to address the challenges of module analysis. Granary is a framework for creating flexible and efficient tools that analyze or debug arbitrary, binary Linux kernel modules. Granary uses dynamic binary translation to dynamically rewrite and comprehensively instrument kernel modules. Granary makes it easy to create flexible tools by supporting context-aware runtime code specialization, and by integrating high-level static analysis information with low-level instruction manipulation. Module analysis tools built on Granary are efficient because of Granary's adoption of a relaxed transparency model, and because of Granary's ability to instrument module code without imposing overhead on non-module kernel code.

\section{Introduction}\label{sec:intro}

Granary is a framework for creating flexible and efficient tools that instrument the Linux kernel. Granary's key novelty is its \emph{runtime code specialization} feature, which enables tools to decide what code to instrument, and how to instrument that code. Flexible specialization allows tools to be picky about instrumentation. For example, a tool can choose to instrument every load and store operation, but only inside nested critical sections that are executing in module code. Granary's runtime code specialization feature addresses two limitations of existing kernel analysis systems: \begin{inparaenum}[i)]
	\item they have focused on instrumenting all code \cite{DRK,btkernel,QEMU}; and
	\item they can only instrument code using a single policy.
\end{inparaenum} 

%Operating system (OS) kernels present an apparently insane environment for a transparent binary instrumentation framework.  For example, hardware interrupts can redirect control flow at any instruction, and interrupt delivery cannot be delayed past the execution of memory instructions that might alter the interrupt delivery state~\cite{DRK}. Preserving interrupt transparency, however, requires that interrupts be delayed until the end of any injected instrumentation instructions.  Together, these requirements make it very difficult to write instrumentation tools.  The insane environment of OS kernels also creates several opportunities for analysis tools. For example, interrupts can be ``tamed'' and put to new uses; privileged hardware features can be used; and the kernel ABI and source code can be relied upon to be a good predictor of module behavior at the kernel/module boundary. We also found that the kernel is largely insensitive to transparency issues, allowing us to relax transparency in exchange for both improved performance and greater visibility into the execution of instrumented code.  This increased visibility is particularly important for analysis and debugging tools.

%The challenges and opportunities of the kernel environment motivated the creation of Granary. Granary is a new binary instrumentation framework for the Linux kernel environment, designed specifically for analyzing kernel modules. Granary targets modules because they represent the bulk of new bugs and new code in development in the Linux kernel \cite{FaultsInLinux}. Granary allows developers to create flexible and dynamic analysis tools for arbitrary Linux kernel module binaries. Granary has two main goals: \begin{inparaenum}[i)]
%	\item make it easy to create efficient, low-level analysis tools; and
%	\item provide high-level static analysis information to the analysis tools.
%\end{inparaenum} Granary achieves the first goal by using dynamic binary translation (DBT) to provide low-level access to instructions, memory, and interrupts. Granary achieves the second goal by giving tools direct access to the results of its own static analysis of the Linux kernel source code. By meeting these goals, Granary tools are able to do things like  figure out which data structure fields are associated with each memory access.

%Granary addresses three limitations of existing program analysis systems: \begin{inparaenum}[i)]
%	\item they have focused on instrumenting all code \cite{DRK,btkernel,QEMU};
%	\item they do not allow for flexible runtime code specialization; and
%	\item they do not provide static type information to the instrumented code.
%\end{inparaenum} 

Instrumenting all code (e.g., the whole kernel) imposes unnecessary performance overheads when only a subset of the code is being analyzed (e.g., modules, functions). In practice, we would like to target instrumentation at specific code for efficiency, while also allowing rich and detailed instrumentation of the targeted code. Limiting the scope of what is instrumented is  important because the overhead introduced by using runtime code instrumentation is usually dominated by the instrumentation itself and not the runtime system supporting that instrumentation.

%This requirement motivates the design of \emph{mixed-mode execution}, which allows Granary tools to instrument only specific code while the majority of code executes with zero overhead. 

%This is challenging because Granary must be comprehensive: all code that is meant to be instrumented should be instrumented. In other words, relinquishing control to execute kernel code natively must not cause Granary to miss the execution of any module code.

Another limitation of existing systems is that they are unable to perform efficient runtime code specialization, i.e. they are unable to change how code is instrumented based on the context in which that code will run. For example, Granary's Read-Copy-Update (RCU) debugging tool, \texttt{RCUdbg}, specializes instrumentation at different stages of use of the RCU synchronization API. Correct usage of the RCU API allows programmers to create scalable, shared data structures that permit concurrent access and modification. The ``trick" of RCU is that readers might not see the latest write to the data structure, and writers must synchronize to modify the data structure. \texttt{RCUdbg} distinguishes between code executing inside \emph{read-side} and \emph{write-side} critical sections, and instruments each code context differently. In read-side critical sections, \texttt{RCUdbg} detects bugs when readers access potentially non-existent data, do not the correct accessor API, or attempt to write to shared data. In write-side critical sections, \texttt{RCUdbg} detects when writers write to shared data without using the correct API. 


%In reader-side code

%For example, Granary has a tool that  dynamically constructs inter- and intra-procedural control flow graphs (CFGs). The outputs of this tool enable Granary to do better runtime register allocation, which improves tool efficiency. Part of building CFGs requires tracking whether or not a given basic block is the entrypoint to a function. This is challenging with existing DBT systems because some optimized functions will re-execute their first basic block, which might incorrectly appear as new function calls in the CFGs. Granary solves this problem using demand-based runtime code specialization: every entrypoint basic block is instrumented in one way, and later re-executions of any entrypoint basic blocks are specialized and instrumented in another way.

The \texttt{RCUdbg} tool is an example of \emph{context-driven code specialization}: code executing inside of a read-side critical section is instrumented differently from code executing inside of a write-side critical section, and code execution outside of both contexts executes with minimal instrumentation. In a Granary tool, each tracked execution context is associated with its own instrumentation policy. An instrumentation policy decides how to instrument code that will execute within the associated context. Granary is able to track execution contexts because it explicitly maintains policy information in translated basic blocks. This avoids typical poblems with tracking context in just-in-time DBT systems (e.g., maintaining runtime context-tracking state, reentrancy, and the inability to predict later execution contexts at translation time).

% In the previous example, executing code is classified into two contexts: entrypoint and non-entrypoint basic blocks.  In the case of our CFG tool, two policies track the two execution contexts: $P_{\mathit{call\_entry}}$ and $P_{\mathit{after\_entry}}$. $P_{\mathit{call\_entry}}$ applies to the first execution of the first basic block of each function, and connects functions in the inter-procedural CFG.  $P_{\mathit{after\_entry}}$ applies to all other executed basic blocks within each function, including re-executions of the first basic block. $P_{\mathit{after\_entry}}$ connects basic blocks in the intra-procedural CFG. 

%Finally, an important missing feature of existing low-level instrumentation systems is the ``big picture'' view that one gets from static analysis information. Similarly, static analysis tools lack access to instructions, memory, and processor-specific behaviors that are only available to low-level runtime instrumentation tools. Existing mixed static/dynamic analysis tools \cite{NaCl,AddressSanitizer,ThreadSanitizer} appear to get the best of both worlds, but sacrifice on runtime flexibility by committing to a single instrumentation policy at a program's compile time. The benefits of high-level static analysis motivated its inclusion into Granary using a technique we call \emph{reifying instrumentation}. Granary statically analyzes the Linux kernel source code to bootstrap learning about  the interfaces that enable modules and the kernel to interact. Granary exposes this information to instrumentation tools in the form of type and function \emph{wrappers}. Tools can use wrappers to apply context-specific policies when instrumenting module code as well as to inspect and manipulate parts of kernel/module memory in a type-safe way.

%For example, we are actively developing a tool that uses static type information to learn about all the fields of all kernel data structures that are read from or written to by modules. This information allows us to model things like how a typical file system operates on kernel data structures in the process of opening a file. From these models, we can construct behavioral patterns about how modules use kernel data structures and functions. These patterns can be used to classify modules \cite{DeviceDriverClassification} and identify spurious behavior \cite{LXFI}.

The rest of the paper describes Granary in more detail. \Cref{sec:dbt} describes how Granary is implemented. 

% \Cref{sec:modes} describes how to efficiently implement mixed-mode execution. \Cref{sec:policies} describes a new method for dynamically changing how code is instrumented. \Cref{sec:reify} describes how to integrate static analysis information into a DBT system. Finally, \Cref{sec:eval} evaluates Granary's performance.

\section{Dynamic Binary Translation}\label{sec:dbt}

Granary uses dynamic binary translation (DBT) to instrument the Linux kernel. DBT is used for emulation \cite{QEMU}, runtime optimization \cite{DynamoRIOOptimization}, and runtime instrumentation (analysis \cite{DynamoRIO, DRK, btkernel, ProfilingSimics}, security \cite{Vx32,NaCl,ProgramShepherding}, and debugging \cite{Valgrind}). We chose DBT because \begin{inparaenum}[i)]
	\item static binary analysis tools are unable to cope with \texttt{x86}'s mix of variably-sized instructions and data; 
	\item some kernel code is only distributed in a binary format (e.g. proprietary device drivers); and
	\item virtualization-based approaches are unable to analyze non-virtualizable device drivers \cite{DRK}.
\end{inparaenum}

%Granary translates and instruments code at one of three default granularities: kernel, module, and function (also called probe-based instrumentation). 
%
%At runtime, the granularity of instrumentation can be altered based on the execution context. For example, a tool analyzing modules for bugs can choose not to instrument a particular function (e.g. whitelist the function) if that function is known to not contain bugs, or known to generate many false-positives.
%
%\paragraph{Kernel}
%Granary interposes on the system call and interrupt entrypoints, located in the x86-64 \texttt{MSR\_LSTAR} model-specific register and interrupt descriptor table (IDT), respectively. 
%
%\paragraph{Module}
%
%
%\paragraph{Function}
%
%Granary can interpose on the execution of specific kernel functions in three ways:
%\begin{enumerate}
%	\item {\bf Wrapping:} Granary can substitute the execution of a kernel function when that function is invoked from instrumented code. A different function can be substituted depending on the context from which the original function is invoked (instrumented kernel code, instrumented module code). We call this type of interposition ``wrapping'' because it is commonly used to inspect the substituted function's arguments and return value, while still invoking the original function.
%	\item {\bf Substituting:} Granary can substitutes the execution of a kernel function, even when that function is executed by native kernel code. This is useful when tools only instrument part of the kernel (e.g. modules or functions) but require visibility on specific kernel events happening outside of the instrumented code (e.g. particular memory allocations).
%	\item {\bf Instrumented Wrapping/Substituting:} Granary can optionally instrument wrapped or substituted code. This is useful when tools want to observe function arguments/return values using high-level C/C++ code, but where that function must be instrumented for visibility.
%\end{enumerate}

Granary translates and instruments kernel binaries one basic block at a time. In Granary, a basic block is a sequence of instructions ending in a conditional branch, \texttt{ret}, or \texttt{jmp}, but not a \texttt{call} instruction. Entrypoint basic blocks (e.g. interrupt vectors, system call entrypoints, module initializers) are translated ahead-of-time, and all other basic blocks are translated just-in-time (JIT) as execution ``discovers'' them. Translated basic blocks are linked together and stored in a globally accessible \emph{code cache}.

Granary's just-in-time translation approach means that code executing from the code cache may yield control to Granary to request the address of the next basic block to execute. When instrumented code yields to Granary, a ``context switch'' occurs that transfers execution to a CPU-private stack where Granary operates. Granary context-switches back to the code cache when the next basic block has been found or translated so that instrumented execution may continue. Similar to other DBT systems, Granary uses caching and hot code patching to reduce the number of context switches. In our experience, kernel code stabilizes very quickly: context switches stop happening after the first few seconds of translation.

Basic blocks in Granary's code cache contain x86-64 binary instructions. Associated with each basic block is meta-data describing the instructions of the basic block, and the instrumentation policy used to instrument those instructions. Basic block meta-data includes: \begin{inparaenum}[i)]
	\item the length in bytes of the original and translated basic blocks;
	\item the address of the first instruction in the original basic block; and
	\item the instrumentation policy.
\end{inparaenum} Basic block meta-data is queried and extended by Granary instrumentation tools. For example, Granary's control-flow graph build tool, \texttt{cfg}, extends the basic block meta-data to track per-block execution counts. Block meta-data is also queried by interrupt handlers when deciding how to handle interrupts and exceptions in instrumented code, and by debuggers (e.g., \texttt{gdb}) to give contextual information about a translated basic block. 

\subsection{Transparency}\label{sec:transparency}

An instrumentation system is transparent if, given the same inputs, the instrumented and uninstrumented versions of a program behave in the same way \cite{Transparency}. Granary includes configurable levels of transparency to address the trade-off between transparency and overhead. By default, Granary uses a relaxed transparency model for efficiency, flexibility and increased visibility, but if instrumentation for some module requires transparency then it can be enabled at the cost of increased overheads and decreased visibility. By default, Granary exposes the following two artifacts: code cache addresses as function and interrupt return addresses.

\paragraph{Function return addresses}\label{para:return_address_transparency} Unlike most DBT frameworks, Granary inlines \texttt{call} instructions into basic blocks, which exposes code cache addresses to instrumented module code in the form of return addresses. Inlining \texttt{call} instructions avoids return-address mispredictions and unnecessary control-flow by building longer basic blocks. 

\paragraph{Interrupt return addresses} Kernel interrupts are not sensitive to return addresses and hence relaxing return address transparency has no effect on interrupts. A special case arises for page fault exceptions that occur within specific kernel functions that access user space data \cite{btkernel}. The Linux kernel records ranges of code addresses that are permitted to fault in an ``exception table'' data structure, and on a page fault exception, checks if the interrupt return address belongs to one of the pre-defined ranges. Granary records any exception table entries as part of each basic block's meta-data, so that a page fault in a basic block can be mapped to the correct native faulting program counter.

%\paragraph{Module function wrappers} Granary does not maintain transparency when its type wrappers replace pointers to module functions with pointers to wrapped module functions. This is perhaps the riskiest break in transparency within Granary because a module might treat a function pointer as being representative of the module being in some particular state.\footnote{We have not yet encountered a module where changing pointers to module functions into module wrappers altered the module's behavior under instrumentation.} If Granary is configured to use transparent module function pointer addresses in shared data structures then fast attaching is disabled and tools will not have access to the same level of static program information.

\subsection{Reentrancy}

\paragraph{Code cache}
Granary's code cache and JIT translation mechanism is fully reentrant. Context-switches into Granary disable interrupts and switch execution onto a CPU-private stack. Context switches out of Granary and into the code cache or native code returns execution to the native stack and re-enables interrupts. This approach presents a minor challenge: context-switches save or restore the entire register state, so as not to disturb the execution of the instrumented program. Therefore, if a context switch into Granary occurs because the next basic block must be found/translated, then the address of that next basic block must be stored \emph{somewhere} in order for execution to eventually reach that basic block when the context switches back to the code cache. But context switching is not allowed to perturb the register state of the instrumented program, and so the target address must be resolved by another means.

Like btkernel, Granary solves this problem of figuring out how to get to where the code needs to go  by generating ``edge'' code for each direct control-flow instruction (e.g. \texttt{call}, \texttt{jmp}, \texttt{jcc}). When a basic block is first translated, its direct control-flow instructions (CTIs) are converted into jumps to CTI-specific edge code. Edge code lives outside the code cache and is responsible for transferring control from a basic block ($B_{pred}$) to Granary so that the next basic block ($B_{succ}$) can be found/translated, and then connected ($B_{pred} \underset{\texttt{jz}}{\to} B_{succ}$) via hot code patching. Reentrancy of edge code is obtained by spilling registers onto the runtime call stack, and later restoring them before jumping back to the now-patched CTI (\Cref{fig:direct_edge_code}).

Granary also generates edge code for indirect control-flow instructions; however, this edge code is split into two separate components: entry and exit edge code. Entry edge code is generated on a per-indirect CTI instruction basis, and spills registers onto the runtim call stack. Entry edge is responsible for indirect branch lookup, and will indirectly jump to the exit edge code for a given basic block. Exit edge code is generated for each basic block that is targeted by an indirect CTI. Exit edge code restores the register state saved by the entry edge code, and then directly jumps to the beginning of the targeted basic block.

\paragraph{Instrumentation}

% Code block showing an example JZ (jump if ZF=1).
\lstset{language=[x64]Assembler}
\newsavebox\nativejcc
\begin{lrbox}{\nativejcc}
\begin{minipage}[b]{4cm}
\begin{lstlisting}[basicstyle=\footnotesize\ttfamily]cmp $0x0, %rax

jz value_is_zero 
...






( fall through )
\end{lstlisting}
\end{minipage}
\end{lrbox}

\newsavebox\translatedjcc
\begin{lrbox}{\translatedjcc}\begin{minipage}[b]{4cm}\begin{lstlisting}[basicstyle=\footnotesize\ttfamily]  cmp $0x0, %rax
jz_to_patch:
  jmp edge_value_is_zero
  ...
edge_value_is_zero:
  ( spill 2 registers )
  ( save value_is_zero )
  ( save target policy )
  call patcher_for_jz
  ( restore 2 registers )
  jmp .jz_to_patch
\end{lstlisting}
\end{minipage}
\end{lrbox}

\begin{figure}[t!]
\subfloat[Native Code]{\usebox\nativejcc}
\hfill
 \subfloat[Translated Code]{\usebox\translatedjcc}
\caption{\label{fig:direct_edge_code}Example translation of a conditional control flow instruction (\texttt{jz}) and its edge code. The edge code saves the intended target of the jump (\texttt{value\_is\_zero}) and the instrumentation policy to apply when translating the targeted code, and then context-switches into Granary by invoking \texttt{patcher\_for\_jz}. The \texttt{patcher\_for\_jz} function atomically replaces the {\bf\texttt{ jmp}}\texttt{ edge\_value\_is\_zero} instruction with a fully resolved \texttt{jz} instruction, and upon return, executes the newly resolved \texttt{jz} instruction.}
\end{figure}

One challenge with any DBT system is how it copes with non-reentrant instrumentation. For example, instrumentation that modifies CPU-private data structures cannot be safely executed by pre-emptive kernel running on a multi-core system. If the task being instrumented is interrupted, then the kernel might decide to re-schedule the task to resume its execution on a different CPU. Any modifications to CPU-private state would lead to undefined behavior.

DRK's solution to this problem assumes that all injected instrumentation instructions are potentially non-reentrant, and \emph{delays} interrupts occuring within instrumented code \cite{DRK}. Interrupt delaying allows DRK to maintain the precise interrupt semantics of x86: interrupts only arrive on \emph{logical} instruction boundaries, and kernel interrupt handlers only observe faults or exceptions at native instruction boundaries. This approach introduces subtle complexity when creating instrumentation that injects instructions both before (pre) and after (post) native instructions. For example, an interrupt occuring in the pre-instrumentation of a native instruction will be delayed only until the native instruction. When the interrupt is handled and execution attempts to return to the interrupted instructions, DRK re-instruments the ``tail" of interrupted basic block, starting from the native instruction. Instrumenting this tail involves injecting pre- and post-instrumentation around the first native instruction. If the pre-instrumentation is not idempotent, then its re-execution will result in undefined behavior. We encountered this issue when using DRK for kernel module instrumentation, where our pre-instrumentation spilled registers to the runtime call stack, and our post-instrumentation restored the spilled registers. Because of interrupt delaying, the spilling code was sometimes repeated many times, whereas the restoring was only executed once. This led to hard-to-debug kernel crashes.

Btkernel provides no solution to the problem of handling non-reentrant instrumentation, beyond requiring that instrumentation tools disable and re-enable interrupts around non-reentrant instrumentation code \cite{btkernel}. Unfortunately, disabling and re-enabling interrupt is costly.

Granary charts a middle ground between the two aforementioned approaches. Interrupt delaying has value because it enables non-reentrant instrumentation to be injected without costly disabling/enabling of interrupts, while not supporting delaying has value because not all instrumentation is non-reentrant. By default, Granary does not delay interrupts. However, instrumentation tools can declare \emph{interrupt delay ranges}: sequences of instructions (which can contain internal control flow) inside of a basic block that must be executed atomically. Granary encodes bounds information about interrupt delay ranges inside of each basic block's meta-data. When a basic block is interrupted, Granary inspects the meta-data and determines if the interrupt must be delayed. If a delay is required, then the instructions belonging to the delay range are copied into a CPU-private buffer, followed by code that re-constructs the interrupt stack frame and re-issues the interrupt to Granary. Granary then emulates a return from the current interrupt handler (without issuing an \texttt{iret} instruction), but leaves interrupts disabled and redirects execution into the copied code. When the copied code finishes, Granary regains control and observes the reconstructed interrupt stack frame. At this point, Granary or the tool can choose to handle the interrupt or defer the handling of the interrupt to the kernel.

%, but introduces implementation complexities and high overheads for interrupt-heavy workloads. Precise interrupt delivery is one of the ways which DRK maintains transparency: interrupt handlers only 
%The interaction between DRK's interrupt delaying mechanism and how also makes designing instrumentation subtlely harder. For example, Granary's \texttt{RCUdbg} tool injects instrumentation inside of read-side critical sections that looks for loads/stores from tainted addresses. When necessary, \texttt{RCUdbg} spills and restores registers around each checked load/store operation, so that the registers can be safely used by the checking code. This is an example of pre and post instruction instrumentation: some instructions are injected before a native instruction, and some instructions are injected after. DRK is oblivious to the relationship between pre/post instrumentation, and so an interrupt that occurs within the pre-instrumentation will only be delayed until the native instruction.

\section{Specializing on What to Instrument}

\begin{figure*}[ht!]
\lstset{language=C, tabsize=2, stepnumber=1}
\begin{multicols}{2}
\begin{lstlisting}[basicstyle=\footnotesize\ttfamily]
struct device_driver {
  ...
  int (*probe)(struct device *);
  int (*remove)(struct device *);
  void (*shutdown)(struct device *);
  int (*suspend)(struct device *, pm_message_t);
  int (*resume)(struct device *);
  ...
  const struct dev_pm_ops *pm;
  ...
};
\end{lstlisting}
\columnbreak
\begin{lstlisting}[basicstyle=\footnotesize\ttfamily]
TYPE_WRAPPER(struct device_driver, {
  PRE_OUT {
    ABORT_IF_FUNCTION_IS_WRAPPED(arg.probe)
    WRAP_FUNCTION(arg.probe);
    WRAP_FUNCTION(arg.remove);
    ...
  }
  POST_OUT {
    POST_WRAP(arg.pm);
  }
})
\end{lstlisting}
\end{multicols}
\caption{The Linux device driver structure is shown on the left. The automatically generated type wrapper for this structure is shown on the right. In the wrapper code, \texttt{arg} is a reference to a \texttt{struct device\_driver} object passed as, or referenced by, an argument to a kernel or module wrapper. Code in the \texttt{PRE\_OUT} section is applied to arguments of the wrapped type before a kernel wrapper is invoked. Similarly, code in the \texttt{POST\_OUT} section is applied to arguments of the wrapped type after a kernel wrapper is invoked. \texttt{POST\_WRAP} invokes the type wrapper that is specific to the value to which it is applied (\texttt{arg.pm}). Type wrappers also support \texttt{\_IN} suffixes instead of \texttt{\_OUT} suffixes, which apply to data going into modules (i.e., over module wrappers). Finally, the \texttt{RETURN\_} prefix is used to apply some code to return values of either kernel or module wrappers.}
\label{fig:type_wrapper}
\end{figure*}

Granary allows tool developers to precisely specify what code should be instrumented. Example granularities of instrumentation include the whole kernel, only specific modules, only specific functions, or on-demand instrumentation based on events (e.g. interrupts or exceptions).

\subsection{Kernel}
Like DRK and btkernel, Granary can instrument the whole kernel. Instrumenting the whole kernel requires interposing on the system call entrypoint (x86-64 \texttt{MSR\_LSTAR} model-specific register) and on the interrupt vector entrypoints (stored in the interrupt descriptor table). For whole-kernel instrumentation, Granary boostraps by duplicating the the early entrypoint instructions that are responsible for switching execution onto a kernel stack, and then injects a jump to translate/instrumented kernel code after the stack switch. The translation process continues when a system call or interrupt occurs and follows the jump from duplicated into translated code. Granary only translate code that is known to operate on a kernel stack so that instrumentation tools can safely save transient data in a reentrant way by spilling that data to the stack.

\subsection{Modules}\label{sec:module}

Granary can target instrumentation only at specific kernel modules, while executing the rest of the kernel natively and without overhead. Instrumenting only specific kernel modules is particularly useful for debugging tools because modules represent the bulk of new kernel code under development, and contain the most bugs \cite{FaultsInLinux}.

Granary interposes on an existing module by using hardware page protection to mark an existing module as non-executable and trap attempts to execute that module's code. Granary transparently recovers from these traps by returning execution to instrumented code. Granary takes control of dynamically loaded modules by interposing on the Linux kernel's module loading process. When a kernel module is loaded, Granary bootstraps by translating the first basic block of the module's initialization function.  It then replaces the pointer to that function with a pointer to the translated basic block. The translation process continues when the kernel initializes the module by invoking the translated module code. As with existing modules, the native code segments of dynamically loaded modules are page protected to prevent and recover from accidental execution. 

While sufficient for comprehensiveness, hardware page protection is an inefficient mechanism for gaining control of module code execution. That is, after the (instrumented) module initializer runs to completion, Granary needs an efficient way to regain control of future invocations of the module's code. Our approach to regaining control is based on the observation that modules tell the kernel about their interfaces by registering functions with the kernel. We expect that at least some of the registered functions will be executed by the kernel because this is the mechanism by which modules extend the kernel's functionality. To Granary, registered module functions represent potential future \emph{attach} points, where a switch from native to instrumented execution will occur. Furthermore, Granary discovers additional attach points at detach points by observing pointers to module functions that are passed by modules to the kernel.

Granary uses static analysis to dynamically discover attach points by wrapping the kernel/module interface. Kernel functions are wrapped by \emph{kernel wrappers} that inspect and traverse argument pointers in search of pointers to module functions. \Cref{fig:type_wrapper} shows an example of a type wrapper for the Linux device driver structure. If a pointer to a module function (a future attach point) is discovered, then Granary replaces that pointer with a function-specific \emph{module wrapper}. Granary inspects and modifies arguments to module wrappers in the same way as for kernel wrappers. This allows Granary to discover kernel entry points that will cause instrumentation to detach. Finally, Granary redirects execution to the appropriate kernel or instrumented module function after wrapping has occurred.

Kernel and module wrappers invoke \emph{type wrappers} to find and wrap pointers to module functions that are directly or indirectly referenced by kernel/module function arguments. A type wrapper is a function that recursively  traverses the in-memory object graph and converts pointers to module code into pointers to wrapped module functions. Type and kernel wrappers are automatically generated at Granary's compile time by scripts that statically analyze the kernel source code. Granary automatically matches any variable in a kernel wrapper to a type wrapper if the base type (absent pointers, specifiers, and qualifiers) of that variable matches the type wrapper's wrapped type. Similarly, Granary automatically generates module wrappers using a combination of compile-time meta-programming and runtime code generation to match type wrappers to the declared arguments of \texttt{C} function pointer types.  Granary's wrapper approach has two benefits: (i) it is more efficient than the trap-based alternative, and (ii) it gives instrumentation tools direct and semantically meaningful access to data crossing the kernel/module boundary.

After Granary bootstraps on the module initialization process, \emph{attaching} occurs in one of three ways: \begin{enumerate}
	\item {\bf Implicit attaching:} the kernel returns to instrumented module code in the code cache. This occurs when instrumented moduel code is interrupted, or when the module invokes a kernel function (detaching) and that function returns to the instrumented module code (attaching).
	\item {\bf Fast attaching:} the kernel invokes a wrapped module function. The first instance of fast attaching for any kernel module is when the kernel's module loader invokes the instrumented module initializer function.
	\item {\bf Slow attaching:} the kernel invokes unwrapped module code. If the kernel executes a module function which was passed to the kernel in a type-unsafe manner, then the processor will raise a fault because Granary uses hardware page protection to prevent module code from being executed. Granary handles these faults by returning execution to the instrumented version of the faulting module code.
\end{enumerate}

Granary \emph{detaches} when control transfers from instrumented code to native (uninstrumented) code. Detaching occurs in one of two ways: \begin{enumerate}
	\item {\bf Implicit detaching:} instrumented module code returns to the kernel code which originally invoked the module, or is interrupted (initial interrupt handling is done by the kernel).
	\item {\bf Wrapped detaching:} instrumented module code invokes a kernel wrapper, which later transfers control to the kernel.
\end{enumerate}

Granary's wrappers expose a wealth of static program information to instrumentation tools. Instrumentation tools can use information present in wrappers to alter how code is instrumented, or to simply inspect the runtime object graph in a type-safe way. For example, we are actively using type information present in wrappers to create a tool that generates models of typical module behavior. The wrapper information serves two roles in our tool. First, we use type wrappers to assign type-specific IDs to module-allocated memory that is shared across the module/kernel interface. Type ID assignment is critical because it allows us to match memory reads and writes to specific kernel data structure fields. Second, we use the kernel and module function wrappers to generate call graphs of a module's execution. This call graph models module behavior (according to the kernel) because we label module code nodes with kernel data structure and function pointer field names (derived from module wrappers). This labelling allows us to generalize across similar modules. For example, code reached by calling the \texttt{ext4\_mount} and \texttt{btrfs\_mount} functions from the \texttt{ext4} and \texttt{btrfs} file system modules, respectively, are both labelled as \texttt{file\_system\_type::mount}, because the function addresses are stored (and later replaced by module wrappers) in the \texttt{mount} field of a \texttt{file\_system\_type} data structure. The combined records from multiple ``trusted'' modules of the same class (e.g., mature, open-source file systems) models the behavior of typical kernel modules. We hope that these models will help us build tools that classify and identify spurious module behavior. 

\subsection{Functions}\label{sec:function_wrapper}

Granary can instrument or not instrument functions in several different ways. For example, \emph{probe-based} instrumentation allows Granary to selectively instrument only a particular function, while \emph{wrapper-based} instrumentation allows Granary to isolate a particular function that is invoked by instrumented code, and choose to instrument that function or run it natively (all while having the flexibility to arbitrarily inspect that function's arguments and return values in a type-safe way).

\paragraph{Probe-based instrumentation}

Probe-based instrumentation allows an instrumentation tool to replace \emph{all} executions of a particular kernel function, regardless of whether or not the replaced function is executed by native or instrumented code \cite{KProbes, AnyWhereAnyTimeDBT, KernInst}. Probe-based instrumentation can be used to wrap a particular function without instrumenting its code, or to wrap and instrument the probed function's code.

Probe-based instrumentation provides visibility into specific kernel events (e.g. memory allocations) to instrumentation tools that don't instrument the whole kernel (e.g. module-only). One example tool that combines module-only and probe-based instrumentation is Granary's kernel module memory leak detector. This leak detector finds leaks of module-owned memory. We say that the some memory is owned by a module iff that memory is deallocated by a module. Unfortunately, ownership of module-allocated memory (for an arbitrary, binary module) is not known \emph{a priori}. Therefore, our tool conservatively assumes that all module-allocated memory is module-owned, unless it is freed by the kernel (observed via probe-based instrumentation of a kernel memory deallocator). We also use probe-based instrumentation to assign module ownership to kernel-allocated \texttt{sk\_buff} data structures, which are allocated by the kernel but deallocated by network device driver modules.

\paragraph{Wrapper-based instrumentation}

Wrapper-based instrumentation makes use of the same wrapper capabilities used by module-only instrumentation (\Cref{sec:module}), but is available regardless of the instrumentation granularity. Wrapper-based instrumentation is similar to probe-based instrumentation, but permits more selectivity in terms of the context in which the function will be wrapped. That is, a function can be wrapped in one way when executed by instrumented kernel code, and in another way when executed by instrumented module code. Finer-grained selectivity can be achieved by having individual instrumentation policies manually decide how to wrap a given function.

\begin{figure}[t!]
\lstset{language=C, tabsize=2, stepnumber=1}
\begin{lstlisting}[basicstyle=\footnotesize\ttfamily]
MODULE_FUNCTION_WRAPPER(__kmalloc, (size, gfp), {
  void *addr = __kmalloc(size, gfp);
  if(is_valid_address(addr)) {
    add_watchpoint(addr, addr, size);
  }
  return addr;
})
\end{lstlisting}
\caption{Example module function wrapper for the \texttt{\_\_kmalloc} function. This wrapper adds an ``address watchpoint" to the address returned by the kernel's \texttt{\_\_kmalloc} memory allocator. Granary's module memory leak detector uses address watchpoints to track the liveness of module-allocated memory. Every access of a ``watched address" marks the associated memory as live within the current epoch. Memory that has not been live for several consecutive epochs is treated as a potential leak, and a is checked for reachability using a heap memory scanner.}
\label{fig:malloc_wrapper}
\end{figure}

Granary's module memory leak detector is one example application that uses the selectivity of wrappers (with respect to their calling context) to distinguish between module-allocated and kernel-allocated memory. \Cref{fig:malloc_wrapper} shows a module function wrapper that adds ``address watchpoints"  \cite{BehaveOrBeWatched} to module-allocated memory.

%Granary can instrument, or not instrument, particular function

\subsection{On-demand}

On-demand instrumentation makes use of Granary's control over interrupt and exception handling. When a specific exception or interrupt occurs, a Granary tool can turn on instrumentation for the interrupted code. This ties in well with Granary's address watchpoints system, which taints memory addresses by changing their high-order 16 bits into a ``non-canonical" form. Watched addresses trap when accessed by native code, which introduces an opportunity for on-demand instrumentation to be enabled. Another application of on-demand instrumentation is data watchpoints based on hardware page protection and memory shadowing.

Turning on instrumentation when specific data is accessed makes on-demand instrumentation inherently \emph{data-centric} as opposed to \emph{code-centric}. That is, instrumentation is targeted at instructions accessing specific data, instead of the typical ``dragnet" approach employed by most DBT systems, which instrument all memory loads/stores so as to capture the select few that access key data.

% instrumentation turns on when specific data is accessed, and can be turned off at the granularity of an instruction, a basic block, the current executing function, or the current executing task/thread. 

%\section{Mixed-Mode Execution}\label{sec:modes}
%Granary supports two modes of execution: instrumented and native. Module code is instrumented and executes from Granary's code cache, which is under Granary's control. Non-module kernel code runs natively. A mode switch occurs when execution transfers between native and instrumented code. Some mode switches happen naturally (e.g., when instrumented code returns to native code) and other mode switches are mediated by Granary (e.g., when instrumented module code invokes a kernel function).
%
%The mode switch from instrumented module code to native kernel code is easier to detect since the instrumented code runs under Granary's control. Granary treats all kernel functions as \emph{detach} points, where a mode switch from instrumented to native code occurs. At these points, Granary stops executing.
%
%However, Granary needs a way to regain control when native kernel code invokes module code. Our fallback solution for regaining control uses hardware page protection to trap attempts by the kernel to execute native module code. We handle these traps by redirecting execution to instrumented module code. While comprehensive, this approach is not ideal because: \begin{inparaenum}[i)]
%	\item trapping on every execution attempt introduces overhead; and
%	\item the trap does not provide sufficient information about which interfaces were being used by the kernel to invoke the module.
%\end{inparaenum}


\section{Specializing on How to Instrument}\label{sec:how}

Granary gives tools the ability to be selective about what is instrumented. A natural extension of this selectivity is to allow tools to specialize how code is instrumented by dynamically switching the policy used to instrument code. Specialization is desirable in several cases. For example, some instrumentation is high-overhead, but does not need to be applied to all code. One example of this is Granary's \texttt{RCUdbg} tool, which instruments read-side and write-side critical sections more heavily than the rest of code. Unfortunately, existing DBT systems fall short in their support for runtime code specialization \cite{DRK,btkernel,Pin,DynamoRIO}. 

To implement the equivalent of \texttt{RCUdbg}'s runtime code specialization using an existing DBT system requires that a tool developer maintain, update, and check runtime state to know whether or not code is executing inside or outside of a read-side critical section. If a basic block is executing in a read-side critical section, then instrumentation should be enabled, and if it's not in a read-side critical section then it should be disabled. Enabling/disabling instrumentation might be done at the instruction granularity, which introduces at least one branch instruction per native instruction, or at the granularity of basic blocks, which requires generating two copies of each native basic block, where a single branch at the beginning of a basic block chooses between the instrumented or uninstrumented version. Up until now, it has been assumed that there is a mechanism for maintaing state that consistently tracks the current execution context of kernel code (inside vs. outside of a read-side critical section). However, maintaining state in a way that is reentrant and resilient again arbitrary pre-emption and resumption is onerous.

%Existing systems move the burden of code specialization to instrumentation tool developers. 

Fortunately for tool developers, Granary has built-in support for runtime code specialization that does not require manual state management or cause code cache bloat. Granary approach to runtime code specialization uses a technique called \emph{context-driven instrumentation}.

%For example, instrumenting only module code allows Granary to specialize the execution of native code. In the same way, policy switching allows Granary's \texttt{RCUdbg} tool to specialize the execution of instrumented code executing inside of read-side and write-side critical sections. In fact, optimizing the performance of an early Granary tool was the original motivation for policy switching.

\subsection{Context-Driven Instrumentation}\label{sec:policies}

%We developed a Granary tool that detects several Read-Copy-Update (RCU) API misuses in module code. Our tool focused on read-side critical sections (delimited by calls to \texttt{rcu\_read\_lock} and \texttt{rcu\_read\_unlock} in the code), however, most code executes outside of read-side critical sections. As an optimization, we wanted our heavyweight API-checking instrumentation to apply only when the code was executing within a read-side critical section. Implementing this optimization was challenging because each basic block was only translated once, and we had no way to know whether or not the heavyweight instrumentation should be applied to it. Knowing the state (within or outside a read-side critical section) at translation time was not sufficient, since the same translated block could later be executed in the opposite context. That is, code executing \emph{within} a read-side critical section may have originally been translated while executing \emph{outside} of a read-side critical section, and would thus omit the heavyweight API-checking instrumentation. This omission could cause our tool to miss bugs (i.e., it would not be comprehensive) when the basic block executes within a read-side critical section.

%To implement this optimization, 

Granary enables specializing instrumentation to the context in which the instrumented code will execute. Specialization is achieved by allowing multiple, differently instrumented versions of the same code to co-exist within Granary's code cache. That is, if the next basic block to execute has not yet been translated/instrumented for the current execution context, then a new version of that basic block will be created  specifically for the the current execution context.



%In the case of our \texttt{RCUdbg}, there were three execution contexts: within and outside of a read-side critical section. If the same module basic block is  executed in the two contexts, then Granary's code cache would contain two different instrumented versions of that basic block. The way that Granary distinguishes between different execution contexts is with \emph{instrumentation policies}. 

Granary tools identify execution contexts by creating instrumentation policies. An instrumentation policy is both a name for an execution context, as well as a function that decides how to instrument basic blocks that will execute within that context. All tools define an initial policy that Granary uses to instrument module code. Tools are not limited to one policy though: any policy can declare a policy switch that will take effect when a selected control-transferring instruction (CTI) is executed. The effect of a tool specifying a policy switch on a CTI is that the basic block(s) targeted by that CTI will be instrumented according to the specified policy. CTIs with unspecified policies inherit their policies from their containing basic blocks.

For example, our \texttt{RCUdbg} invokes a policy switch from policy $P_{\mathit{null}}$ to policy $P_{\mathit{read\_critical}}$ when $P_{\mathit{null}}$'s instrumentation function observes a \texttt{call} instruction to the \texttt{rcu\_read\_lock} Linux kernel function. A similar switch occurs from $P_{\mathit{read\_critical}}$ back to $P_{\mathit{null}}$ at a \texttt{rcu\_read\_unlock}. Specialization that distinguishes between the inside and outside of read-side critical sections allows \texttt{RCUdbg} to do heavy-weight checks on functions like \texttt{memset}, but only when \texttt{memset} is invoked from within a read-side critical section.

Because policies name an execution context, they also represent states in a finite state machine. That is, code cache execution is in the state named by a policy if the  executing code was instrumented by that policy. A state transition occurs when control transfers from code instrumented by one policy to code instrumented by another policy. A limitation with this approach is that a single state does not encode the sequence of previous states that led to execution being in the current state. For example, RCU permits nested read-side critical sections. If two read-side critical sections are nested then switching from $P_{\mathit{read\_critical}}$ to $P_{\mathit{null}}$ on the first \texttt{call} to \texttt{rcu\_read\_unlock} meant that our tool would lose track of being in the context of the outer read-side critical section.  \texttt{RCUdbg} solves this problem by tracking the nesting depth of RCU read-side critical sections using policies, e.g. $P_{\mathit{read\_critical}_1}$, $P_{\mathit{read\_critical}_2}$, etc.

An alternative and sometimes more powerful model for policy switching is to prevent policy switching on function return instructions (e.g. either through manually specifying a policy switch, or by inheriting the policy from the basic block containing the \texttt{ret} instruction). In most cases, function returns are natural policy reverting points. This is because the current instruction pointer (program counter) encodes the current execution context (by being in a basic block instrumented by some policy), and the code cache return addresses stored on the runtime call stack encode all previous execution contexts. This asymmetry between \texttt{call}s and \texttt{ret}s yields a more powerful context-tracking state machine: \texttt{call}s place contextual breadcrumbs (in the form of return addresses) on the runtime call stack, and \texttt{ret}s read these breadcrumbs to return to a previous context. Policy switches under this lens behave similarly to state transitions in a pushdown automaton; \texttt{call} instructions push a new state onto the stack for each function, \texttt{ret} instructions pop the current function's state from the stack, and other CTIs induce state transitions within the current function by altering control flow.

Granary implements policy tracking and switching by encoding policy information into the meta-data and CTIs of basic blocks. \Cref{fig:direct_edge_code} shows an example of how policy information is directly recorded into control-transfer edge code. If a policy switch is not specified on an instrumented CTI then that CTI inherits the policy used to instrument the basic block containing the CTI. When an instrumented CTI executes for the first time, it yields control to Granary with the CTI target and policy information as inputs. Granary decodes and instruments the targeted instructions according to the input policy information. Because Granary's translation mechanism depends only on CTI policy and target information, Granary is able to ensure that policy information is never lost or corrupted, even in the face of concurrent executions of the same module code, arbitrary pre-emption, and arbitrary resumption.




\section{Application: RCUdbg}

Read-Copy-Update (RCU) is a scalable, lock-free synchronization mechanism used in the Linux kernel \cite{RCU,RCUInLinux}. When correctly used, RCU permits concurrent reads and writes to shared data structures, but requires that writers synchronize among themselves. Like other synchronization mechanisms, RCU has a deceptively simple API. Misuses of the API can result in hard-to-find bugs that can be implementation-specific (i.e. bugs that manifest in one version of RCU but not another) or architecture-specific (e.g. bugs that only exhibit on architectures with a relaxed memory model). For example, RCU bugs sometimes manifest as use-after-free bugs, which can result in silent data corruption (if the underlying memory remains used) or kernel panics. 



% (assuming that pointer reachability is sufficient to determine semantic reachability in this case).

Granary's \texttt{RCUdbg} tool is designed to catch RCU bugs related to kernel code misusing pointers to RCU-protected data structures. The following is a non-exhaustive list of requirements that must be met by reader and writer threads operating on an RCU-protected data structure. The list omits a description of how RCU-protected data structures are garbage collected, which is beyond the scope of the bugs caught by \texttt{RCUdbg}.

%the associated memory may or may not be free, but the data structure occupying that memory can't be proven to exist. This 

%  However, this class of RCU bugs is not well-suited to being detected by a use-after-free memory checker tool
%RCU is notoriously hard to use, despite its relatively simple API. RCU usage bugs, like bugs with other synchronization mechanisms, often stay hidden and only exhibit themselves for very specific thread interleavings/schedules.



% go silently unnoticed because the memory 


%While bugs related to mutual exlusion locks might result in easily observable deadlocks, bugs related to RCU typically result in use-after-free errors, which sometimes go silently unnoticed, sometimes result in kernel panics, and most of the time work as expected because the associated data is .

\begin{enumerate}
	\item Readers must tolerate an inconsistent view of the shared data structure. That is, a reader operating concurrently with a writer might observe either the version of the data before or after the writer's changes have been published. This can be complicated when writers perform multiple, discrete updates to a data structure. One example of this is a doubly linked list: a writer that is adding a new entry into a doubly linked list must update at least two RCU-protected pointers.
	
	\item Readers must operate within \emph{read-side} critical sections, which are delimited by \texttt{rcu\_read\_lock} and \texttt{rcu\_read\_unlock}, respectively.
	
	\item Within read-side critical sections, readers can only read a shared data structure after obtaining a pointer to that shared data structure using the \texttt{rcu\_dereference} API function. An RCU-protected pointer is like a normal pointer, but points to a potentially incomplete or partially-constructed data structure. \texttt{rcu\_dereference} operates on an RCU-protected pointer and returns an ``unprotected" version of the pointer. The reader can use the unprotected pointer to safely access the shared data structure within the remainder of the its read-side critical section. If \texttt{rcu\_dereference} is not used, then the reader might observe a partially-constructed (i.e. incomplete) version of the shared data structure. Notably, accessing the original data structure and not the dereferenced data structure pointer--even after an \texttt{rcu\_dereference}--is undefined behavior. That is, within a read-side critical section, RCU only guarantees the ``completeness" of data structures pointed to by RCU-dereferenced pointers. Data accessed within a read-side critical section is \emph{not} guaranteed to exist outside the read-side critical section.

	\item Writers must synchronize among themselves before any writes to a shared data structure are performed. Synchronization is typically achieved via mutual exclusion locks, spin locks, or by having a single, designated writer thread. We say writes to the shared data structure occur within \emph{write-side} critical sections.

	\item RCU only defines how writers can update RCU-protected pointer within a shared data structure. Modification of non-RCU-protected pointer fields within an RCU-protected data structure is beyond the scope of RCU's correctness guarantees.

	\item Writers wishing to update an RCU-protected pointer in a shared data structure must use the \texttt{rcu\_assign\_pointer} API function. \texttt{rcu\_assign\_pointer} assigns a new value to an RCU-protected pointer in a shared data structure.

	\item Readers must access any pointer field operated on by an \texttt{rcu\_\linebreak[0]assign\_pointer} using \texttt{rcu\_\linebreak[0]dereference}. For example, if the \texttt{next} pointer field in \emph{any} list element of an RCU-protected list has been assigned to by \texttt{rcu\_assign\_\linebreak[0]pointer}, then \emph{all} \texttt{next} pointer fields must be accessed by readers of that list by using \texttt{rcu\_\linebreak[0]dereference}. In this sense, using \texttt{rcu\_\linebreak[0]dereference} and \texttt{rcu\_assign\_\linebreak[0]pointer} marks specific variables and data type fields as containing RCU-protected pointers.

\end{enumerate}

% NOTE!!!!
%
% All of those spaces before /* BUG! */ and such are so that the lstlisting will be *close* to the columnwidth!! Somehow linewidth param of lstset didn't work for me :-(

\lstset{
	language=C,
	tabsize=2,
	stepnumber=1,
	morekeywords={rcu_read_lock,rcu_dereference,rcu_read_unlock,rcu_assign_pointer,spin_lock,spin_unlock}}

% Example code showing an existence bug.
\newsavebox\rcuexistencebug
\begin{lrbox}{\rcuexistencebug}
\begin{lstlisting}[basicstyle=\footnotesize\ttfamily]
rcu_read_lock();
  elm = rcu_dereference(list_head);
  data = elm->data;                  /* Valid */
rcu_read_unlock();
data = elm->data;                     /* BUG! */
\end{lstlisting}
\end{lrbox}

% Example code showing another existence bug.
\newsavebox\rcuderefbug
\begin{lrbox}{\rcuderefbug}
\begin{lstlisting}[basicstyle=\footnotesize\ttfamily]
rcu_read_lock();
  elm = rcu_dereference(list_head);
  if(elm != NULL)
    data = list_head->data;           /* BUG! */
rcu_read_unlock();
\end{lstlisting}
\end{lrbox}

% Example code showing another existence bug.
\newsavebox\rcuassignderefbug
\begin{lrbox}{\rcuassignderefbug}
\begin{lstlisting}[basicstyle=\footnotesize\ttfamily]
spin_lock();
  elm = kmalloc(...);
  elm->next = list_head->next;
  rcu_assign_pointer(list_head->next, elm);
spin_unlock();
...
rcu_read_lock();
  elm = rcu_dereference(list_head);
  while(elm != NULL) {
    data += elm->data;
    elm = elm->next;                  /* BUG! */
  }
rcu_read_unlock();
\end{lstlisting}
\end{lrbox}

\begin{figure}[ht!] %
\subfloat[\label{fig:rcu_existence_bug}Despite correctly obtaining a pointer to a shared data structure using \texttt{rcu\_dereference} within a read-side critical section, properties (2) and (3) tell us that the data pointed to by \texttt{elm} is not guaranteed to exist after the read-side critical section ends.]{\fbox{\usebox\rcuexistencebug}} %

\subfloat[\label{fig:rcu_deref_bug}Despite the call to \texttt{rcu\_dereference}, properties (1) and (3) tell us that \texttt{list\_head} is not guaranteed to be equivalent to \texttt{elm} because a writer might have concurrently changed the value of the \texttt{list\_head} pointer. That is, if \texttt{elm} is non-\texttt{NULL}, then the data pointed to by \texttt{elm} is guaranteed to exist for the remainder of the read-side critical section. However, the value of \texttt{list\_head} is \emph{not} guaranteed to remain consistent--even within read-side critical sections--and so \texttt{list\_head} might point to a \texttt{NULL} pointer when \texttt{elm} does not.]{\fbox{\usebox\rcuderefbug}} %

\subfloat[\label{fig:rcu_assign_bug}This example shows the code for a writer thread adding a new element into the second position of an RCU-protected list. The code for reader threads is also shown, where readers iterate over the list by traversing \texttt{next} pointers and sum up the \texttt{data} fields of each visited list element. The access of \texttt{elm->next} violates of properties (3) and (7). That is, the next element of the list is only guaranteed to be fully constructed within the read-side critical section if the reader executes \texttt{rcu\_dereference(elm->next)}. Therefore, the first read of \texttt{elm->next} might yield a pointer to partially constructed data.  Because that data is not guaranteed to be fully constructed, the next read of \texttt{elm->next} (i.e. \texttt{elm->next->next}) might read an invalid pointer from uninitialized memory. Treating the value of uninitialized memory as a valid pointer could trigger a kernel panic or silent data corruption.]{\fbox{\usebox\rcuassignderefbug}} %
\caption{\label{fig:rcu_bugs}Example RCU bugs related to misuses of RCU-protected pointers.}
\end{figure}

\texttt{RCUdbg} is primarily concerned with finding bugs related to RCU-rpotected pointer misuses by reader threads (both inside and outside of read-side critical sections). As mentioned above, RCU bugs sometimes manifest as use-after-free bugs. Unfortunately, a pure use-after-free-based checking approach is insufficient (\Cref{fig:rcu_assign_bug} is one such example). \Cref{fig:rcu_existence_bug} is an example of an RCU usage bug that can manifest as a use-after-free bug if one is unlucky enough to observe a very specific thread schedule. Suppose that execution is interrupted after \texttt{rcu\_read\_unlock}, but before the access to \texttt{elm->data}. If \texttt{elm} is garbage collected and freed before the code resumes its execution, then the access to \texttt{elm->data} is a use-after-free bug. This kind of bug is a time bomb waiting to go off. Fortunately, \texttt{RCUdbg} catches the bug in \Cref{fig:rcu_existence_bug} right when \texttt{elm->data} is accessed.


%\Cref{fig:rcu_bugs} shows some example usages of RCU, and some of the more subtle pointer-related bugs that can happen because of the interaction of rules (1), (2), (3), and (7).


\texttt{RCUdbg} detects the the following RCU bugs related to misuses of RCU-protected pointers.
\begin{itemize}[leftmargin=3.2em]
	\item[B1)] Reads through an RCU-dereferenced pointer outside of a read-side critical section (\Cref{fig:rcu_existence_bug}).
	\item[B2)] Reads through an RCU-dereferenced pointer in the wrong read-side critical section.

	\item[B2)] Reads through an RCU-assigned pointer that don't first use \texttt{rcu\_dereference} (\Cref{fig:rcu_deref_bug}).
	\item[B3)] Reads through fields of RCU-protected data structures that should be accessed via \texttt{rcu\_dereference} (\Cref{fig:rcu_assign_bug}).
	
	\item[B5)] Writes through an RCU-dereferenced pointer.
	\item[B6)] Writes through an RCU-assigned pointer by a non-write thread.
\end{itemize}

Bugs B1 and B2 relate to whether or not we can prove that a given RCU protected data structure exists. Bugs B2 and B3 relate to whether or not a given RCU protected data structure has been fully constructed. Finally, bugs B5 and B6 relate to invalid uses RCU-protected and RCU-unprotected pointers, where correct synchronization among writers is not maintained.

% or is constructed at a given program point. Bugs B2 through B

% of this class of RCU bugs is proof-of-existence or proof-of-constructedness bugs. That is, an RCU-protected data structue exists if it is \emph{reachable} (i.e. by at least one reader thread), and if it is \emph{fully constructed} (i.e. all fields or components of the data structure have been initialized). 

\texttt{RCUdbg} gains visibility on RCU operations in two ways: \begin{enumerate}
	\item The Linux kernel RCU implementation is annotated to provide more information to \texttt{RCUdbg}. This information includes source code file and line information for bug reporting, and additional memory references that \texttt{RCUdbg} uses to get better visibility on bugs. These annotations are strictly internal to the RCU API, and thus require no changes to code using this API.
	\item The annotated RCU API functions are specialized using Granary's function wrapping feature (\Cref{sec:function_wrapper}). Specifically, 
\end{enumerate}

%\texttt{RCUdbg} interposes on an annotations are added into the Linux kernel's RCU implementation. These annotations 


%uses Granary's address watchpoints feature \cite{BehaveOrBeWatched} to ``taint" addresses participating in RCU-protected data structures. 

%Function wrappers are used to specialize the execution of \texttt{rcu\_\linebreak[0]dereference} and \texttt{rcu\_assign\_\linebreak[0]pointer}:

\begin{figure}[t!]
\lstset{language=C, tabsize=2, stepnumber=1}
\begin{lstlisting}[basicstyle=\footnotesize\ttfamily]
// As used in the code:
//   p = rcu_dereference(q);
// Internally annotated to become:
//   p = __rcudbg_deref(&q, q);
// Wrapper:
FUNCTION_WRAPPER(__rcudbg_deref, (q_addr, q), {
  if(is_watched_address(q_addr)) {
    // Make sure q_addr
  } else {

  }
  return add_watchoint(q, DEREF);
})
\end{lstlisting}
\caption{}
\label{fig:rcu_dereference_wrapper}
\end{figure}

%\begin{center}
%\begin{minipage}{0.6\columnwidth}
%\begin{lstlisting}[basicstyle=\footnotesize\ttfamily]
%elm = rcu_dereference(list_head);
%\end{lstlisting}
%\end{minipage}
%\end{center}
%
%Becomes:
%
%\begin{center}
%\begin{minipage}{0.8\columnwidth}
%\begin{lstlisting}[basicstyle=\footnotesize\ttfamily]
%elm = deref_taint(rcu_dereference(list_head));
%\end{lstlisting}
%\end{minipage}
%\end{center}
%
%Tainted addresses are introduced in two ways:
%\begin{description}
%	\item[\texttt{rcu\_assign\_pointer}] operates on two pointers: 
%	\item[\texttt{rcu\_dereference}] operates on a potentially tainted pointer (as introduced by \texttt{rcu\_assign\_\linebreak[0]pointer}, and returns a deref-tainted pointer.
%\end{description}
%

%Unfortunately, approaching RCU bug detection from the perspective of finding use-after-free errors is insufficient. Use-after-free detection attacks a symptom of the problem that is highly dependent on thread interleavings and scheduling. That is, RCU can be misused in a way that results in very unlikely, but potential use-after-free bugs. An example of this case is when a writer thread, operating on a RCU-protected linked list, unlinks an element and queues its memory for later freeing. RCU provides a mechanism whereby readers can safely access this element for a brief period of time (called a read-side critical section). When the memory is eventually freed depends is implementation-defined, but depends on other reader activity.

%How soon after a read-side critical section ends 

% There are many cases readers can access an RCU-protected data structure because they have not been freed \emph{yet}


%

% a data structure cannot be proven to exist at some potentially allocated memory. Existence in this case relates to semantic reachability. For example, if a list element $E_n$ from an RCU-protected list is reachable from the head of the list by traversing zero-or-more intermediate elements $E_0, ..., E_{n-1}$, then $E_n$ is said to exist. While related, memory leak detection (e.g. via conservate mark and sweep) is also insufficient for ladidadidaaaa TODO TODO TODO

\section{Evaluation}\label{sec:eval}

What has been evaluated:

\paragraph{lmbench}
\begin{itemize}
	\item Btkernel
	\item Granary, full-kernel, null
	\item Granary, full-kernel, even/odd
	\item Granary, full-kernel, even/odd, where even/odd is a runtime check and branch.
\end{itemize}

\paragraph{fileserver}
Setup: 1GB RAM, 3GB partition, ext3+jbd, >1GB mounted file system.
\begin{itemize}
	\item full-kernel
	\begin{itemize}
		\item Btkernel
		\item Granary, null
		\item Granary, even/odd
		\item Granary, even/odd, where even/odd is a runtime check and branch.
	\end{itemize}
	\item module-only, (ext3+jbd)
	\begin{itemize}
		\item Granary, null
		\item Granary, even/odd
		\item Granary, even/odd, where even/odd is a runtime check and branch.
	\end{itemize}
\end{itemize}

\paragraph{kernel compile}
Same as fileserver, but with 8GB RAM instead of 3GB.

\paragraph{recursive copy of kernel source tree}
Same as fileserver, but with 8GB RAM instead of 3GB.

\section{Related Work}\label{sec:related}
Prior work can be classified as either whole-kernel/system DBT, or probe-based kernel instrumentation. We discuss how Granary differs from these approaches.
\paragraph{Whole-kernel/system DBT}
PinOS \cite{PinOS} is a port of the Pin \cite{Pin} DBT framework that performs whole-system instrumentation. It has high overheads and depends on hardware virtualization, which prevents it from instrumenting non-virtualizable kernel modules. 

QMEU is a flexible, whole-system emulator with kernel-level paravirtualization support (using KVM) \cite{QEMU}. Like PinOS, QEMU cannot instrument non-virtualizable device drivers.

DRK \cite{DRK} is a kernel space port of the DynamoRIO \cite{DynamoRIO} DBT framework. DRK instruments the entire kernel and all device drivers/modules. DRK follows a strict transparency model, which limits the flexibility of instrumentation and increases overheads. 

Btkernel is a fast and scalable system for the kernel \cite{btkernel}. Like Granary, but unlike DRK, btkernel applies a relaxed transparency model when instrumenting the kernel. For exmaple, code cache addresses are exposed to the instrumented kernel in the form of function call return addresses and interrupt return addresses. Unlike Granary, btkernel operates using an instruction translation ``rulebook", which limits instrumentation to being defined ahead-of-time, on a per-instruction basis. Btkernel does not support any form of runtime code specialization. That is, like DRK, btkernel instruments all code, but cannot change the granularity of instrumentation or the instrumentation policy applied to code.

TODO: JIFL

\paragraph{Probe-based Instrumentation} Several systems support injecting code at specific locations within kernel code.

TODO: SystemTap

KernInst \cite{KernInst} can inject code at almost any location in an unmodified commodity OS. By default, KernInst uses debugging information normally present in kernel binaries to specify probe-points; however, absolute memory addresses can also be provided.

Other examples with similar or more restricted functionality include LTTng \cite{LTTng}, KProbes \cite{KProbes}, DProbes \cite{DProbes}, and ftrace \cite{ftrace}. These systems are unable to perform fine-grained instrumentation of instructions or memory.


\section{Conclusion}\label{sec:conclusion}
%We created Granary to address the challenges of instrumenting binary kernel modules.  Granary provides mixed-mode execution to remove the overheads of DBT for uninstrumented kernel code.  A relaxed transparency model further improves performance and allows greater visibility into the interactions between module and kernel code.  Granary also supports policy-driven instrumentation, which allows tools to specialize their instrumentation based on the context in which code executes.  Finally, Granary exposes high-level static analysis information to dynamic instrumentation tools, making it easy to match low-level memory reads with specific fields in data structures at the source code level. 

%Together, Granary's features simplify the development of powerful kernel module analysis tools, while delivering lower overheads than previous kernel dynamic binary instrumentation solutions.



\bibliographystyle{acm}
\bibliography{library}

\end{document}
