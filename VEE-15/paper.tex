\documentclass[preprint]{sigplanconf}
\usepackage{cleveref,amsmath,chngcntr}
\usepackage{textcomp} % for tilde
\usepackage{paralist} % for in-paragraph lists
\usepackage{url}
\usepackage{graphicx}
\usepackage{multicol,multirow}
\usepackage{tabularx}

% For drawing FSMs
%\usepackage{tikz}
%\usetikzlibrary{arrows,automata}

% Used for code snippets
\usepackage{listings,courier}
\usepackage{subfig,epsfig}
\DeclareCaptionType{copyrightbox}
\usepackage{epsfig}


% Settings on code listings.
\lstset{language=C,
		xleftmargin=0pt,
		xrightmargin=0pt,
		framexbottommargin=0pt,
        framextopmargin=0pt,
        framesep=0pt}
\usepackage{enumitem}

% Add listing language for assembly
\lstdefinelanguage
   [x64]{Assembler}     % add a "x64" dialect of Assembler
   [x86masm]{Assembler} % based on the "x86masm" dialect
   % with these extra keywords:
   {morekeywords={CDQE,CQO,CMPSQ,CMPXCHG16B,JRCXZ,LODSQ,MOVSXD, %
                  POPFQ,PUSHFQ,SCASQ,STOSQ,IRETQ,RDTSCP,SWAPGS, %
                  rax,rdx,rcx,rbx,rsi,rdi,rsp,rbp, %
                  r8,r8d,r8w,r8b,r9,r9d,r9w,r9b}} % etc.


% For leaving some comments in the draft.
\newcommand{\comment}[1]{}

% Customize cleverref
\crefname{section}{Section}{Sections}

\begin{document}

%make title bold and 14 pt font (Latex default is non-bold, 16 pt)
\title{Virtual Register Scheduling}

%for single author (just remove % characters)
\authorinfo{Peter Goodman \and Angela Demke Brown \and Ashvin Goel}
{University of Toronto}{}
% end authorinfo

\maketitle
\subsection*{Abstract}

A common challenge faced by static and dynamic binary instrumentation
systems is how to manage transient state used by injected or modified binary
instructions. Virtual register systems solve this problem by providing
instrumentation tool writers with an infinite set of registers for use by
instrumented code. Unfortunately, generic virtual register systems are hard
to implement in practice. In this work, we present a generic virtual
register system for a complex instruction set (x86-64). Our virtual register
system addresses three limitations of prior work: % \cite{Pin,Valgrind}
\begin{inparaenum}[i)]
	\item they require that all code be comprehensively instrumented;
	\item they require a strong notion of threads and the availability of
 thread-local storage, and;
	\item they require a runtime monitor, and therefore cannot be applied
by both static and dynamic binary instrumentation systems.
\end{inparaenum}

%\section{Introduction}\label{sec:intro}




% 

%Existing virtual register implementations  punt on generality and require
%\begin{inparaenum}[i)]
%	\item that all code be comprehensively instrumented,
%	\item that the instruction set or OS be virtualized, and
%	\item the availability of dedicated, thread-local storage.
%\end{inparaenum} These re

%In this work, we present a generic virtual register system for a complex instruction set (x86-64). Our virtual register system is general in that it
%\begin{inparaenum}[i)]
%	\item does not require that all code be instrumented,
%	\item works on user and kernel space,
%	\item operates directely on x86-64 binary instruction,
%	\item does not assume the availability of 
%\end{inparaenum}


%\section{Related Work}\label{sec:related}

%\section{Evaluation}\label{sec:eval}

%\section{Conclusion}\label{sec:conclusion}



%\bibliographystyle{acm}
%\bibliography{library}

\end{document}
